\documentclass[../article.tex]{subfiles}

\begin{document}

\section{The DFT and other nonsense}
Our Purpose in this section is to give the reader an understanding of the theoretical framework which gives rise to the Discrete Fourier Transform.

\subsection{Preliminary Definitions and Theorems}


\begin{definition}
Let $\mathbb{Z}(N)$ be the set of all N-th roots of unity. Then,
 \[Z(N) = \{z \in C : z^N = 1 \}\]
\end{definition}
\begin{theorem}
For all $N \in \mathbb{Z}$, we have that
\[\mathbb{Z}(N)= \left\{1, e^{(2 \pi i)/N}, e^{(4 \pi i)/N}, \ldots, e^{(2(N-1) \pi i)/N}\right\}\]
\end{theorem}

\begin{proof}
Let $z$ $\in \mathbb{Z}(N)$. Then, $z^N = (r \cdot e^{i \theta})^N = 1$. Therefore, $\mid r^N \mid \mid e^{i N \theta} \mid = 1$. Hence, $r = e^{i N \theta} = 1$. Therefore, $N\theta = 2k\pi$, where $k \in \mathbb{Z}$. Let $\zeta$ = $e^{2 \pi i /(N-1)}$. Then,
\begin{align*}
	(\zeta^0)^N &= e^0 = 1 \\
	(\zeta^1)^N &= e^{2 \pi i} = 1 \\
	(\zeta^2)^N &= e^{4 \pi i} = 1 \\
	{} &\vdotswithin{=}\\
	(\zeta^{N-1})^N &= (e^{2 \pi i(N-1)/N})^N \\
	{} &= (e^{2 \pi i}e^{-2 \pi i/N})^N \\
	{} &= 1 \cdot e^{-2 \pi i} \\
	{} &= 1
\end{align*}
Therefore, $\zeta^k \in \mathbb{Z}(N)$ for all $k$ such that $0 \leq k \leq N-1$. Notice that the set $Q = \{0, 1, 2, \ldots, N-1\}$ is a residue system modulo $N$, and that the set $\zeta^Q = \left\{1, e^{2 \pi i/N}, e^{4 \pi i/N}, \ldots, e^{2(N-1) \pi i/N}\right\} \subseteq \mathbb{Z}(N) $.
Now, for every $n \in \mathbb{Z}$, there exists an integer $m \in Q$ such that $n \equiv m \mod N$, since $Q$ is a complete residue system modulo $N$. When $k \geq N-1$, we have that $k \equiv l \mod N$ for some $l \in Q$. Therefore, $\mathbb{Z}(N) \subseteq Q$, and since we know $Q \subseteq Z(N)$, we have $Q = Z(N)$.
\end{proof}

\begin{theorem}
$\mathbb{Z}(N)$ is a finite abelian group.
\end{theorem}

\begin{proof}
That $| \mathbb{Z}(N)| < \infty$ is self-edvident. Thus, $\mathbb{Z}(N)$ is a finite group. Letting $z,w \in \mathbb{Z}(N)$, we find that $z \cdot w = e^{2 \pi i (k+l)/N}$ for some $k,l \in \mathbb{Z}$. Then, by the complete residue theorem of number theory, there exists an $m \in \{1, 2, 3, ..., N-1\}$ such that $(k+l) \equiv m \mod N$. Hence, $e^{2 \pi i (k+l)/N} = e^{2 \pi i m/N} \in \mathbb{Z}(N)$. Therefore, $\mathbb{Z}(N)$ is closed under complex multiplication.

Let $z,w \in \mathbb{Z}(N)$. Then $z \cdot w = e^{2 \pi i k/N} \cdot e^{2 \pi i l/N} = e^{2 \pi i (k+l)/N} = e^{2 \pi i (l+k)/N} = e^{2 \pi i l/N} \cdot e^{2 \pi i k/N} = w \cdot z$. Therefore $(\mathbb{Z}(N), \cdot)$ is commutative.

Next, let $z,w,s \in \mathbb{Z}(N)$. Then, $(z \cdot w) \cdot s = (e^{2 \pi i k/N} \cdot e^{2 \pi i l/N}) \cdot e^{2 \pi i t/N} = e^{2 \pi i ((k+l)+t)/N} = e^{2 \pi i (k+(l+t))/N} = e^{2 \pi i k/N} \cdot (e^{2 \pi i l/N} \cdot e^{2 \pi i t/N}) = z \cdot (w \cdot s)$. So $(\mathbb{Z}(N), \cdot)$ is associative.

The identity element is $1 \in \mathbb{Z}(N)$, since for all $z \in \mathbb{Z}(N)$ we have $z \cdot 1 = z$.

If $z \in \mathbb{Z}(N)$, then $(1/z)^N = 1/ z^N = 1$. Therefore, $1/z \in \mathbb{Z}(N)$ and the inverse of any $z \in \mathbb{Z}(N)$ is $1/z$.

Therefore, $\mathbb{Z}(N)$ is an abelian group.
\end{proof}

\begin{definition}
For all $x \in \mathbb{Z}$, let $R_N (x) = \{y \in \mathbb{Z} : y \equiv x   \mod   N\}$, where $N \in \mathbb{Z}$
\end{definition}
\begin{theorem}
For all $x,y \in \mathbb{Z}$, $R_N (x) + R_N (y) = R_N (x+y).$
\end{theorem}
\begin{proof}
Let $\alpha \in R_N (x)$ and $\beta \in R_N (y)$. Then, $\alpha = x +kN$ for some $k \in \mathbb{Z}$ and $\beta = y + lN$ for some $l \in \mathbb{Z}$. Then, $(x+y)+(k+l)N \in R_N (x+y)$. Conversely, let $x+y \in R_N (x+y)$. Then, $x = \alpha -kN$ and $y = \beta -lN$. Therefore, $(x+y) = (\alpha + \beta) -(k+l)N$. Therefore, $(x+y) \equiv \alpha +\beta) \mod N$. Therefore, $x \equiv \alpha \mod N$ and $y \equiv \beta \mod N$. Therefore, $x+y \in R_N (x) + R_N (y)$.
\end{proof}

\begin{theorem}
If $F$ is a function on $\mathbb{Z}(N)$, then
\begin{equation}
F(k) = \sum_{n=0}^{N-1} a_n e^{2 \pi nk/N}
\end{equation}
where
\begin{equation}
a_n = \frac{1}{N} \sum_{k=0}{N-1} F(k) e^{-2 \pi ikn/N}.
\end{equation}
\end{theorem}
\begin{proof}
Let $\zeta^{lk} = e^{2 \pi i lk/N}$, where $k,l = 0,1,...N-1$. Then, the set
\begin{equation}
V = \left\{ \frac{1}{\sqrt{N}}, \frac{1}{\sqrt{N}}e^{2 \pi i k/N}, \ldots, \frac{1}{\sqrt{N}}e^{2 \pi i k(N-1)/N} \right\}
\end{equation}
forms an orthogonal basis for $\mathbb{Z}(N)$. So,
\begin{equation}
F(k) = \sum_{n=0}^{N-1} \Big(F(k), \frac{1}{\sqrt{N}}\zeta^{nk} \Big) \frac{1}{\sqrt{N}}\zeta^{nk}
\end{equation}
where $\big(F,G \big)$ represents the Hermitian product of $F$ and $G$. By substitution we obtain
\begin{equation}
\begin{split}
F(k) &= \sum_{n=0}^{N-1} \Bigg(\sum_{k=0}^{N-1} F(k) \frac{1}{\sqrt{N}} \zeta^{-lk} \Bigg) \frac{1}{\sqrt{N}} \zeta^{lk} \\
&= \sum_{n=0}^{N-1} \Bigg(\frac{1}{N} \sum_{k=0}^{N-1} F(k) \zeta^{-lk} \Bigg) \zeta^{lk} \\
&= \sum_{n=0}^{N-1} a_n e^{2 \pi nk/N}
\end{split}
\end{equation}
where
\begin{equation}
a_n = \frac{1}{N} \sum_{k=0}^{N-1} F(k) e^{-2 \pi ikn/N}
\end{equation}
and $e^{2 \pi i/N} = \zeta$.
\end{proof}

\subsection{Legendere Chi Function and Hurwitz Zeta Function}
\begin{definition}
The Legendere Chi Function is defined by
\begin{equation}
\chi_{\nu}(z) = \sum_{k=0}^{\infty} \frac{z^{k+1}}{(2k+1)^{\nu}},
\end{equation}
where $|z| \leq 1$ and $\nu = 2,3,4,\ldots$
\end{definition}

\begin{theorem}
If $\omega^r = e^{2\pi i r/t}$, where $r=1,2,\ldots,t$, then $\chi_{\nu}(\omega^r)$ is $t$-periodic. That is, $\chi_{\nu}(\omega^{r+t}) = \chi_{\nu}(\omega^r)$.
\end{theorem}

\begin{proof}
Let $\omega^r = e^{2\pi i r/t}$. From the definition of $\chi_{\nu}(\omega^r)$, we know
\begin{equation}
\chi_{\nu}(\omega^{r}) = \sum_{k=0}^{\infty} \frac{e^{2 \pi i r(k+1)/t}}{(2k+1)^{\nu}}.
\end{equation}
So,
\begin{equation}
\begin{split}
\chi_{\nu}(\omega^{r+t}) &= \sum_{k=0}^{\infty} \frac{e^{2 \pi i (r+t)(k+1)/t}}{(2k+1)^{\nu}} \\
&= \sum_{k=0}^{\infty} \frac{e^{2 \pi i r(k+1)/t}e^{2 \pi i(k+1)}}{(2k+1)^{\nu}} \\
&= \sum_{k=0}^{\infty} \frac{e^{2 \pi i r(k+1)/t}}{(2k+1)^{\nu}} \\
&= \chi_{\nu}(\omega^r)
\end{split}
\end{equation}

Therefore, $\chi_{\nu}(\omega^r)$ has period $t$ with respect to $r$.
\end{proof}

\begin{corollary}
$(2t)^{\nu} \chi_{\nu}(\omega^r)$ has period $t$ with respect to $r$.
\end{corollary}

\begin{proof}
We have $(2t)^{\nu} \chi_{\nu}(\omega^{r+t}) = (2t)^{\nu} \chi_{\nu}(\omega^r)$ by $Theorem$ 5. Therefore, $(2t)^{\nu} \chi_{\nu}(\omega^r)$ has period $t$ with respect to $r$.
\end{proof}

\begin{definition}
The Hurwitz zeta function is defined by
\begin{equation}
\zeta(s,q) = \sum_{n=0}^{\infty} \frac{1}{(q+n)^s}
\end{equation}
where $s > 1$ and $q > 0$.
\end{definition}

\begin{theorem}
$(2t)^\nu \chi_{\nu}(\omega^{r})$ is the Fourier transform pair of $\zeta(\nu, (2s-1)/2t)$, where $\omega = e^{2\pi i/t}$
\end{theorem}

\begin{proof}
If $\omega = e^{2\pi i/t}$, then $\omega^r = e^{2 \pi i r/t}$ and
\begin{equation}
\begin{split}
\chi_{\nu}(\omega^{r}) &= \sum_{k=0}^{\infty} \frac{(\omega^r)^{2k+1}}{(2k+1)^{\nu}} \\
&= \sum_{k=0}^{\infty} \frac{(e^{2 \pi i r/t})^{2k+1}}{(2k+1)^{\nu}} \\
&= \sum_{k=0}^{\infty} \frac{e^{2 \pi i r(2k+1)/t}}{(2k+1)^{\nu}} \\
&= \sum_{k=0}^{\infty} \frac{e^{(4 \pi irk +2\pi i r)/t}}{(2k+1)^{\nu}} \\
&= \sum_{k=0}^{\infty} \frac{e^{(4 \pi irk +4 \pi ir - 2\pi i r)/t}}{(2k+1)^{\nu}} \\
&= \sum_{k=0}^{\infty} \frac{e^{(4 \pi irk +4 \pi ir)/t}e^{- 2\pi i r/t}}{(2k+1)^{\nu}} \\
&= e^{- 2\pi i r/t} \sum_{k=0}^{\infty} \frac{e^{4 \pi i r(k+1)/t}}{(2k+1)^{\nu}}
\end{split}
\end{equation}

Therefore,

\begin{equation}
\begin{split}
\chi_{\nu}(\omega^r)\omega^r &= \sum_{k=0}^{\infty} \frac{e^{2 \pi i r(2k+1)/t} \cdot e^{2 \pi i r/t}}{(2k+1)^{\nu}} \\
&= \sum_{k=0}^{\infty} \frac{e^{4 \pi i r(k+1)/t}}{(2k+1)^{\nu}} \\
\end{split}
\end{equation}

Now, for every $k \in \mathbb{Z}_{+}$ and fixed integer $t \in \mathbb{Z}_{+}$ there exist $unique$ integers $m$ and $s$ such that $k = mt +s$, where $0 \leq s < t-1$; i.e., $s \in R_{0}(t) = \{0, 1, ..., t-1\}$, where $R_{0}(t)$ is a complete residue system mod $t$. Therefore, by the property of complete residue systems mod $t$, for every integer $k \in \mathbb{Z}_{+}$, there exists an integer $s \in R_{0}(t)$ such that $k \equiv s \mod t$. Let $f(k,t)$ be some function of $k$ and $t$. By our preceding argument, there exist unique $m$ and $s$ in $\mathbb{Z}_{+}$ such that $k = mt +s$. Therefore,

\begin{equation}
\begin{split}
\sum_{m=0}^{\infty} \sum_{s=0}^{t-1} f(tm+s,t) &= f(0 \cdot t + 0, t) + f(0 \cdot t +1, t) + \cdots + f(0 \cdot t + t-1, t) \\
&+ f(1 \cdot t + 0, t) + f(1 \cdot t + 1,t ) + \cdots + f(1 \cdot t + t-1, t) \\
&+ f(2 \cdot t + 0, t) + f(2 \cdot t +1, t) + \cdots + f(2 \cdot t + t-1, t) \\
&+ \cdots \\ta
&= f(0, t) + f(1, t) + \cdots + f(t-1, t) \\
&+ f(t, t) + f(t+1, t) + \cdots + f(2t-1, t) \\
&+f(2t, t) + f(2t +1, t) + \cdots f(3t-1, t) \\
&+\cdots \\
&= \sum_{k=0}^{\infty} f(k,t). \\
\end{split}
\end{equation}

Therefore,
\begin{equation}
\begin{split}
\chi_{\nu}(\omega^r)\omega^r &= \sum_{k=0}^{\infty} \frac{e^{4 \pi ir((k)+1)/t}}{(2(k)+1)^{\nu}} \\
&= \sum_{m=0}^{\infty} \sum_{s=0}^{t-1}  \frac{e^{4 \pi ir((tm+s) +1)/t}}{(2(tm+s)+1)^{\nu}} \\
&= \sum_{m=0}^{\infty} \sum_{s=1}^{t}  \frac{e^{4 \pi ir(tm+s)/t}}{(2tm+2s-1)^{\nu}} \
\end{split}
\end{equation}

Again, let $f(tm + s, t)$ be an arbitrary function. Then the sum
\begin{equation}
\begin{split}
\sum_{s=1}^{t} \sum_{m=0}^{\infty} f(tm+s, t) &= \sum_{s=1}^{t} f(s,t) + f(t+s,t) + f(2t +s, t) + \cdots \\
&= f(1,t) + f(t+1,t) + f(2t+1,t) + \cdots \\
&+ f(2,t) + f(t+2,t) + f(2t+2,t) + \cdots \\
&+ \cdots \\
&+ f(t,t) + f(2t,t) +f(3t, t) + \cdots
\end{split}
\end{equation}

Summing vertically across the arrangement above gives
\begin{equation}
\begin{split}
&f(1, t) + f(2, t) + \cdots + f(t, t) \\
&+ f(t+1, t) + f(t+2, t) + \cdots + f(2t, t) \\
&+f(2t+1, t) + f(2t +2, t) + \cdots f(3t, t) \\
&+\cdots \\
&= \sum_{k=1}^{\infty} f(k,t)
\end{split}
\end{equation}

Therefore,
\begin{equation}
\sum_{s=1}^{t} \sum_{m=0}^{\infty}  f(tm+s, t) = \sum_{k=1}^{\infty} f(k,t) = \sum_{m=0}^{\infty} \sum_{s=1}^{t} f(tm+s,t)
\end{equation}

So our sum
\begin{equation}
\begin{split}
\sum_{m=0}^{\infty} \sum_{s=1}^{t} \frac{e^{4 \pi ir(tm+s)/t}}{(2tm+2s-1)^{\nu}} &= \sum_{s=1}^{t} \sum_{m=0}^{\infty} \frac{e^{4 \pi ir(tm+s)/t}}{(2tm+2s-1)^{\nu}} \\
&= \frac{1}{(2t)^{\nu}} \sum_{s=1}^{t} \sum_{m=0}^{\infty} \frac{e^{4 \pi imr} \cdot e^{4 \pi r is/t}}{(m+\frac{2s-1}{2t})^{\nu}}. \\
&= \frac{1}{(2t)^{\nu}} \sum_{s=1}^{t} \sum_{m=0}^{\infty} \frac{e^{4 \pi r is/t}}{(m+\frac{2s-1}{2t})^{\nu}}
\end{split}
\end{equation}

Therefore,

\begin{equation}
\begin{split}
\omega^{-r}\chi_{\nu}(\omega^r)\omega^r &= \frac{\omega^{-r}}{(2t)^{\nu}} \sum_{s=1}^{t} \sum_{m=0}^{\infty} \frac{e^{4 \pi r is/t}}{(m+\frac{2s-1}{2t})^{\nu}} \\
&=\frac{1}{(2t)^{\nu}} \sum_{s=1}^{t} e^{4 \pi ris/t - 2 \pi ri/t} \sum_{m=0}^{\infty} \frac{1}{(m+\frac{2s-1}{2t})^{\nu}}. \\
&= \frac{1}{(2t)^{\nu}} \sum_{s=1}^{t} e^{2 \pi ir(2s-1)/t} \sum_{m=0}^{\infty} \frac{1}{(m+\frac{2s-1}{2t})^{\nu}} \\
&= \frac{1}{(2t)^{\nu}} \sum_{s=1}^{t} e^{2 \pi ir(2s-1)/t} \zeta(\nu, (2s-1)/2t)
\end{split}
\end{equation}

So,

\begin{equation}
\omega^{-r}\chi_{\nu}(\omega^r)\omega^r = \chi_{\nu}(\omega^r) = \frac{1}{(2t)^{\nu}} \sum_{s=1}^{t} e^{2 \pi ir(2s-1)/t} \zeta(\nu, (2s-1)/2t)
\end{equation}

By Theorem 4, since $\zeta(\nu, (2s-1)/2t)$ is a function on $\mathbb{Z}(t)$, we have that
\begin{equation}
\zeta(\nu, (2s-1)/2t) = \sum_{r=1}^{t} a_{n} e^{2 \pi i r(2s-1)/t},
\end{equation}
where
\begin{equation}
a_n = \frac{1}{t} \sum_{s=1}^{t} \zeta(\nu, (2s-1)/2t) e^{-2 \pi r(2s-1)/t}
\end{equation}
So,
\begin{equation}
\begin{split}
\zeta(\nu, (2s-1)/2t) &= \sum_{r=1}^{t} a_{n} e^{2 \pi i r(2s-1)/t} \\
&= \sum_{r=1}^{t} \Bigg(\frac{1}{t} \sum_{s=1}^{t} \zeta(\nu, (2s-1)/2t) e^{-2 \pi r(2s-1)/t}\Bigg)e^{2 \pi i r(2s-1)/t} \\
&= \frac{1}{t} \sum_{r=1}^{t} (2t)^{\nu} \Bigg(\frac{1}{(2t)^{\nu}} \sum_{s=1}^{t} e^{2 \pi ir(2s-1)/t} \zeta(\nu, (2s-1)/2t \Bigg) e^{-2 \pi ir(2s-1)/t} \\
&= \frac{1}{t} \sum_{r=1}^{t} (2t)^{\nu} \chi_{\nu}(e^{2 \pi ir(2s-1)/t})e^{-2 \pi ir(2s-1)/t} \\
&= \frac{1}{t} \sum_{r=1}^{t} (2t)^{\nu} \chi_{\nu}(\omega^{r(2s-1)})\omega^{-r(2s-1)} \\
&= \frac{1}{t} \sum_{r=1}^{t} (2t)^{\nu} \chi_{\nu}(\omega^{r})\omega^{-r},
\end{split}
\end{equation}

since $r(2s-1) \equiv r \mod t$ implies that $\omega^{r(2s-1)} = \omega^{r}$ by Theorem 1. Similarly, $\omega^{-r(2s-1)} = \omega^{r}$ since $-r(2s-1) \equiv r \mod t$.
\end{proof}

We have shown that
\begin{equation}
\chi_{\nu}(\omega^r) = \frac{1}{(2t)^{\nu}} \sum_{s=1}^{t} e^{2 \pi ir(2s-1)/t} \zeta(\nu, (2s-1)/2t)
\end{equation}
and
\begin{equation}
\zeta(\nu, (2s-1)/2t) = \frac{1}{t} \sum_{r=1}^{t} (2t)^{\nu} \chi_{\nu}(\omega^{r})\omega^{-r}
\end{equation}

Therefore, there exists a discrete Fourier transform $\mathcal{F}$ such that $\mathcal{F}(\zeta(\nu, (2s-1)/2t)) = \chi_{\nu}(\omega^r)$ and $\mathcal{F}^{-1}(\chi_{\nu}(\omega^r)) = \zeta(\nu, (2s-1)/2t)$.

\subsection{Polylogarithm}

\begin{definition}
The polylogarithm is defined by
\begin{equation}
Li_{s}(z) = \sum_{k=1}^{\infty} \frac{z^k}{k^s}
\end{equation}
where $s > 1$ and $|z| < 1$.
\end{definition}

\begin{theorem}
$\chi_{s}(z) = \frac{1}{2} (Li_{s}(z) - Li_{s}(-z))$
\end{theorem}
\begin{proof}
We have
\begin{equation}
\frac{1}{2} (Li_{s}(z) - Li_{s}(-z)) = \frac{1}{2} \sum_{k=1}^{\infty} \frac{z^{k} - (-z)^k}{k^s}
\end{equation}
When $k$ is even the summand is null; when k is odd, the summand is $2z^{k}/k^s$. Therefore,
\begin{equation}
\begin{split}
\frac{1}{2} (Li_{s}(z) - Li_{s}(-z)) &= \frac{1}{2} \sum_{k=1}^{\infty} \frac{z^{k} - (-z)^k}{k^s} \\
&= \frac{1}{2} \sum_{k=1,3,5,...} \frac{2(z)^k}{k^s} \\
&= \sum_{k=0}^{\infty} \frac{z^{2k+1}}{(2k+1)^s} \\
&= \chi_{s}(z)
\end{split}
\end{equation}
\end{proof}
\begin{corollary}
The preceding gives us a unique representation of $\zeta(\nu)$ in terms of $\chi_{\nu}(\omega^r)$. Let $\frac{2s-1}{2t} = 1$. Then, $t = s - \frac{1}{2}.$ Therefore,
\begin{equation}
\begin{split}
\zeta(\nu, \frac{2s-1}{2t}) &= \zeta(\nu) \\
&= \frac{1}{t} \sum_{r=1}{t} (2t)^{\nu} \chi_{\nu}(\omega^{r+t}) \omega^{-r(2t-1)} \\
&= \frac{1}{t} \sum_{r=1}{t} (2t)^{\nu} \chi_{\nu}(\omega^{r+t}) \omega^{r} \\
&= \prod_{p} \frac{1}{1- p^{-\nu}}\qed
\end{split}
\end{equation}
\end{corollary}

\end{document}
